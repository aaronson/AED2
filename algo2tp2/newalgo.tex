% $Id: algo2symb.tex,v 1.6 2004/08/13 15:09:52 fpscha Exp $
\documentclass[a4paper]{article}
\usepackage[a4paper,margin=2cm,top=2.0cm,bottom=2.34cm]{geometry}
\usepackage[spanish,activeacute]{babel}
\usepackage{algo2symb}
\usepackage{newalgo}

\parskip=1.5ex
\pagestyle{empty}

\begin{document}
\RCS $Revision: 1.2 $
\RCSdate $Date: 2006/09/30 14:08:52 $

\begin{center}{\Large Package \texttt{newalgo}} \par\medskip Versi'on \version -- \RCSDate\end{center}
\medskip

Este paquete provee el ambiente \emph{algorithm} y algunas macros para escribir algoritmos en el lenguaje de dise�o de la materia Algoritmos y Estructuras de Datos II (FCEyN, UBA). Los docentes de la c'atedra lo utilizamos para componer pr'acticas, apuntes y parciales; desde luego, los alumnos que decidan usar \LaTeX\ durante la cursada tambi'en est'an invitados a aprovecharlo.

Las l�neas de los algoritmos se numeran para facilitar la referencia a las mismas durante los c�lculos de complejidad.

\bigskip

\noindent \textbf{Listado de comandos provistos}
\medskip

\begin{tabular}{l@{\hspace{36pt}}c}
\verb|\restriccion{a \leq b}| & \restriccion{a \leq b} \\
\verb|\comentario{Esto es un comentario}| & \comentario{Esto es un comentario} \\
\verb|\param{inout}{c}{conj\dealfa}| & \param{inout}{c}{conj\dealfa} 
\end{tabular}

\medskip

Dentro del ambiente algorithm se pueden utilizar los ambientes IF, WHILE, FOR y otros similares, y dentro de IF, el comando \verb|\ELSE|. El comando \verb|\=| dentro de un algoritmo produce la flechita de asignaci�n $\leftarrow$. El comando \verb|\VAR| sirve para declarar las variables a utilizar.

El ambiente algorithm requiere 3 par�metros: nombre del algoritmo, argumentos, y valor de retorno. Si el valor de retorno es vac�o, el algoritmo no devolver� ning�n valor (ver segundo ejemplo).

\medskip

\begin{large}\textbf{Ejemplos:}\end{large}

\restriccion{\forall i, j: nat\ (0\leq i<j<tam(pajar)\Rightarrow pajar[i]\leq pajar[j])}
\begin{algorithm}{B�squedaBinaria}{\param{in}{aguja}{nat}, \param{in}{pajar}{arreglo(nat)}}{res\!: bool}
    \VAR{desde, hasta, actual\!: nat}\\
    desde \= 0 \\
    hasta \= tam(pajar) \\
    res \= false \\
    \begin{WHILE}{desde < hasta \land \lnot res}
      actual \= (desde + hasta) / 2 \comentario{Tomamos el elemento del medio.}\\
      \begin{IF}{pajar[actual] = aguja}
        res \= true
        \ELSE \begin{IF}{pajar[actual] > aguja}
          hasta \= actual
          \ELSE
          desde \= actual + 1 \comentario{Ac'a el elemento encontrado es menor que el que buscamos.}
          \end{IF}
      \end{IF}
   \end{WHILE}
\end{algorithm}

\begin{algorithm}{ReemplazarApariciones}{\param{in}{aguja}{\alfa}, \param{in}{reemplazo}{\alfa}, \param{inout}{pajar}{arreglo\dealfa}}{}
	\VAR{actual\!: nat}\\
	\begin{FOR}{actual\=0 \TO tam(pajar)-1}
		\begin{IF}{pajar[actual]=aguja\comentario{Si encontramos el elemento, lo reemplazamos.}}
			pajar[actual] \= reemplazo
		\end{IF}
	\end{FOR}
\end{algorithm}
\end{document}

